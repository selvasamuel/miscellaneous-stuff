\documentclass{article}

\usepackage{amsmath}						% For the aligned environment
\usepackage[margin=1in]{geometry}
\usepackage{nccmath}						% For the fleqn environment
\usepackage[hyphens]{url}					% For \url. The hyphens option allows line breaks after hyphens in URLs.

\newcommand{\explnspacei}{\hspace{1.59in}}		% Space to be added before explanation in equations
\newcommand{\explnspaceii}{\hspace{3.04in}}
\newcommand{\explnspaceiii}{\hspace{1in}}
\newcommand{\explnspaceiv}{\hspace{3.61in}}

\allowdisplaybreaks

\begin{document}

\noindent
This integral appears in problem 62 on page 360 of \textit{A Treatise on the Integral Calculus, Volume II}, by Joseph Edwards.
\\
\par\noindent
$\begin{aligned}
\int_0^4 \frac{\ln{x}}{\sqrt{4x - x^2}} \, dx & = \int_0^4 \frac{\ln{x}}{\sqrt{4 - 4 + 4x - x^2}} \, dx \\		% \mathrm{d}x
& = \int_0^4 \frac{\ln{x}}{\sqrt{4 - (x^2 - 4x + 4)}} \, dx \\
& = \int_0^4 \frac{\ln{x}}{\sqrt{4 - (x - 2) ^ 2}} \, dx \\
& = \int_{-\pi / 2}^{\pi / 2} \frac{2 \ln{(2 + 2 \sin \theta)}}{\sqrt{4 - 4 \sin^2 \theta}} \cos \theta \, d\theta & \explnspacei x - 2 = 2 \sin \theta \\
& = \int_{-\pi / 2}^{\pi / 2} \ln{(2(1 + \sin \theta))} \, d\theta & \explnspacei \text{Simplification} \\
& = \int_{-\pi / 2}^{\pi / 2} (\ln{2} + \ln{(1 + \sin \theta)}) \, d\theta \\
& = \int_{-\pi / 2}^{\pi / 2} \ln 2 \, d\theta + \int_{-\pi / 2}^{\pi / 2} \ln{(1 + \sin \theta)} \, d\theta \\
& = \theta \ln 2 \biggr\rvert_{-\pi / 2}^{\pi / 2} + \int_{-\pi / 2}^{\pi / 2} \ln{(1 + \sin \theta)} \, d\theta \\	% \big, \bigr, \Bigr and \Biggr are also available.
& = \pi \ln 2 + \int_{-\pi / 2}^{\pi / 2} \ln{(1 + \sin \theta)} \, d\theta
\end{aligned}$
\\ \\ \\
\par\noindent
$\begin{aligned}
I_1 & = \int_{-\pi / 2}^{\pi / 2} \ln{(1 + \sin \theta)} \, d\theta \\
& = \int_{\pi / 2}^{-\pi / 2} -\ln{(1 - \sin \phi)} \, d\phi & \explnspaceii \theta & = - \phi \\
& = \int_{-\pi / 2}^{\pi / 2} \ln{(1 - \sin \phi)} \, d\phi \\
& = \int_{-\pi / 2}^{\pi / 2} \ln{(1 - \sin \theta)} \, d\theta & \explnspaceii \phi & = \theta
\end{aligned}$
\\ \\
\par\noindent
% I switched to using the fleqn and alignat* environments below to allow page break.
% You also need to specify \allowdisplaybreaks which I have done above.
{
\setlength{\abovedisplayskip}{0pt}
\begin{fleqn}
\begin{alignat*}{2}
2I_1 & = \int_{-\pi / 2}^{\pi / 2} \ln{(1 + \sin \theta)} \, d\theta + \int_{-\pi / 2}^{\pi / 2} \ln{(1 - \sin \theta)} \, d\theta \\
& = \int_{-\pi / 2}^{\pi / 2} (\ln{(1 + \sin \theta)} + \ln{(1 - \sin \theta)}) \, d\theta \\
& = \int_{-\pi / 2}^{\pi / 2} \ln{\{(1 + \sin \theta)(1 - \sin \theta)\}} \, d\theta \\
& = \int_{-\pi / 2}^{\pi / 2} \ln{(1 - \sin^2 \theta)} \, d\theta \\
& = \int_{-\pi / 2}^{\pi / 2} \ln{\cos^2 \theta} \, d\theta \\
& = \int_{-\pi / 2}^{\pi / 2} 2 \ln{\cos \theta} \, d\theta \\
& = 2 \int_{-\pi / 2}^{\pi / 2} \ln{\cos \theta} \, d\theta
\end{alignat*}
\end{fleqn}
}
\\
\par\noindent
$\begin{aligned}
I_1 = \int_{-\pi / 2}^{\pi / 2} \ln{\cos \theta} \, d\theta
\end{aligned}$
\\ \\ \\
\par\noindent
$\begin{aligned}
I_2 & = \int_{0}^{\pi / 2} \ln{\cos \theta} \, d\theta \\
& = \int_{\pi / 2}^{0} -\ln{\cos \left( \frac{\pi}{2} - \psi \right)} \, d\psi & \explnspaceiii \theta & = \frac{\pi}{2} - \psi \\		% https://www.overleaf.com/learn/latex/Brackets_and_Parentheses
& = \int_{0}^{\pi / 2} \ln{\cos \left( \frac{\pi}{2} - \psi \right)} \, d\psi \\
& = \int_{0}^{\pi / 2} \ln \sin \psi \, d\psi \\
& = \int_{0}^{\pi / 2} \ln \left( 2 \sin \frac{\psi}{2} \cos \frac{\psi}{2} \right) \, d\psi \\
& = \int_{0}^{\pi / 2} \left( \ln 2 + \ln \sin \frac{\psi}{2} + \ln \cos \frac{\psi}{2} \right) \, d\psi \\
& = \int_{0}^{\pi / 2} \ln 2 \, d\psi + \int_{0}^{\pi / 2} \ln \sin \frac{\psi}{2} \, d\psi + \int_{0}^{\pi / 2} \ln \cos \frac{\psi}{2} \, d\psi \\
& = \psi \ln 2 \biggr\rvert_{0}^{\pi / 2} + \int_{\pi / 2}^{\pi / 4} -2 \ln \sin \left( \frac{\pi}{2} - \theta \right) \, d\theta + \int_{0}^{\pi / 2} \ln \cos \frac{\psi}{2} \, d\psi & \explnspaceiii \frac{\psi}{2} & = \frac{\pi}{2} - \theta \\
& = \frac{\pi}{2} \ln 2 + 2 \int_{\pi / 4}^{\pi / 2} \ln \cos \theta \, d\theta + \int_{0}^{\pi / 2} \ln \cos \frac{\psi}{2} \, d\psi \\
& = \frac{\pi}{2} \ln 2 + 2 \int_{\pi / 4}^{\pi / 2} \ln \cos \theta \, d\theta + \int_{0}^{\pi / 4} 2 \ln \cos \theta \, d\theta & \explnspaceiii \frac{\psi}{2} & = \theta \\
& = \frac{\pi}{2} \ln 2 + 2 \int_{\pi / 4}^{\pi / 2} \ln \cos \theta \, d\theta + 2 \int_{0}^{\pi / 4} \ln \cos \theta \, d\theta \\
& = \frac{\pi}{2} \ln 2 + 2 \left( \int_{0}^{\pi / 4} \ln \cos \theta \, d\theta + \int_{\pi / 4}^{\pi / 2} \ln \cos \theta \, d\theta \right) \\
& = \frac{\pi}{2} \ln 2 + 2 \int_{0}^{\pi / 2} \ln \cos \theta \, d\theta \\
& = \frac{\pi}{2} \ln 2 + 2 I_2
\end{aligned}$
\\ \\ \\
\par\noindent
$\begin{aligned}
I_2 = -\frac{\pi}{2} \ln 2
\end{aligned}$
\\ \\
\par\noindent
{
\setlength{\abovedisplayskip}{0pt}
\begin{fleqn}
\begin{alignat*}{2}
I_3 & = \int_{-\pi / 2}^{0} \ln \cos \theta \, d\theta \\
& = \int_{\pi / 2}^{0} -\ln \cos \psi \, d\psi & \explnspaceiv \theta & = -\psi \\
& = \int_{0}^{\pi / 2} \ln \cos \psi \, d\psi \\
& = \int_{0}^{\pi / 2} \ln \cos \theta \, d\theta & \explnspaceiv \psi & = \theta \\
& = I_2 \\
& = -\frac{\pi}{2} \ln 2
\end{alignat*}
\end{fleqn}
}
\par\noindent
$\begin{aligned}
\int_{0}^{4} \frac{\ln{x}}{\sqrt{4x - x^2}} \, dx & = \pi \ln 2 + \int_{-\pi / 2}^{\pi / 2} \ln (1 + \sin \theta) \, d\theta \\
& = \pi \ln 2 + I_1 \\
& = \pi \ln 2 + \int_{-\pi /2 }^{\pi / 2} \ln \cos \theta \, d\theta \\
& = \pi \ln 2 + \int_{-\pi / 2}^{0} \ln \cos \theta \, d\theta + \int_{0}^{\pi / 2} \ln \cos \theta \, d\theta \\
& = \pi \ln 2 + I_3 + I_2 \\
& = \pi \ln 2 - \frac{\pi}{2} \ln 2 - \frac{\pi}{2} \ln 2 \\
& = 0
\end{aligned}$

\begin{thebibliography}{9}

\bibitem{stack_overflow_solution}
\url{https://math.stackexchange.com/questions/1066006/evaluate-int-04-frac-ln-x-sqrt4x-x2-mathrm-dx/1078278}

\end{thebibliography}

\end{document}

% References:
% 1. https://tex.stackexchange.com/questions/207397/left-aligned-equations
% 2. https://tex.stackexchange.com/questions/403331/how-to-remove-unnecessary-spacing-in-flalign
% 3. https://www.overleaf.com/learn/latex/Brackets_and_Parentheses
